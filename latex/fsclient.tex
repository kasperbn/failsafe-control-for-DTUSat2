\chapter{FSClient}
\label{chap:fsclient}

This chapter will deal with the requirements specification, analysis, design and non-trivial implementation details of FSClient.

\section{Requirements Specification}
The requirements for FSClient are:
\begin{itemize}
	\item must be installed on the staff's desktop computers
	\item must be a unix program
	\item must be a TCP client to FSServer
	\item must forward failsafe commands from other unix scripts to the server
	\item must have an interactive mode
\end{itemize}
In addition to these requirements, the following requirements was identified when the overall design was agreed upon:
\begin{itemize}
	\item must have a data\_only mode
	\item must have an auto\_lock mode
\end{itemize}

\subsection{Analysis and design}
Lets go through the requirements one by one. For each requirement we will enhance the design to incorporate the requirement.

\textbf{Must be installed on the staff's desktop computers} \\
Trivial constraint.

\textbf{Must be a unix program} \\
Trivial constraint.

\textbf{Must be a TCP client to FSServer} \\
Trivial constraint

\textbf{Must forward failsafe commands from other unix scripts to the server} \\
Fsclient will take a failsafe command and its arguments as parameters, send it over TCP to the server and print the response to standard out.

\textbf{Must have an interactive mode} \\
In interactive mode, it prompts the user for a failsafe command, sends it to the server, retreives the response, prints the response and prompts the user again until the user enters exit. The --interactive option will do this.

\textbf{Must have a data\_only option} \\
Sometimes we don't want to see the entire response message but just the data field. The \texttt{--data-only} option will do this.

\textbf{Must have an auto\_lock option} \\
There should also be an option for auto locking when starting the interactive mode. The \texttt{--auto-lock} option will do this.

\subsection{Implementation}
This section deals with the non-trivial implementation details.

There are no contraints to program speed so the FSClient has been implemented in ruby.
Fsclient uses rubys standard command "\$stdin.gets" to prompt the user for input.
The TCP connection has been implemented with eventmachine.
