\chapter{Requirements Specification}
\label{chap:requirements_specification}
A good way to evaluate the success of a project is to state some measurable requirements that can be tested for when the system has been been implemented. This chapter deals with the requirements of the project.
\\ \\
Before stating the requirements they must first be identified. The requirements are directly dictated by or based on meetings with my supervisor who is a member of the DTUSat-2 staff.
\\ \\
The requirements to this project are:
\begin{itemize}

	\item The user must be able to operate the satellite in failsafe mode from a desktop computer via a console program or a GUI.

	\item Command combinations
	\begin{itemize}
		\item	must be able to combine several commands in conditionals and loops.
		\item must be able to save, load and execute these combinations
		\item	must be flexible to create and change
	\end{itemize}

	\item Console program
	\begin{itemize}
		\item must take a command and its arguments as parameters on the command line
		\item must have an interactive mode. This mode must prompt the user for a command, execute it, print the result and prompt again.
	\end{itemize}

	\item GUI
	\begin{itemize}
		\item must show a graphical representation of the satellites health status
		\item must create, save, load, export and execute command combinations
		\item must provide a graphical tool to put together command sequences (without conditionals and loops)
		\item must be easy to install and run
	\end{itemize}

	\item Upload script
	\begin{itemize}
		\item must take a file and a memory address as parameters and upload to the satellite at the given memory address.
		\item must notify the user with the progress
	\end{itemize}

	\item Protocol layer
	\begin{itemize}
		\item must use the protocol layer so that the datalink implementation can be changed later on
	\end{itemize}

\end{itemize}
