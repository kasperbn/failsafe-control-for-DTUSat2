\chapter{User Guide}
\label{appendix:A}
\markboth{\appendixname\ \thechapter}{User Guide}
Contains installation and operation instructions for the individual subsystems.

\section{FSServer}
\textbf{Installation instructions on Ubuntu} \\
FSServer needs ruby and some additional rubygems to run. Here are the commands to install the server.
\begin{verbatim}
	sudo apt-get install ruby1.8 ruby1.8-dev rubygems1.8
	sudo gems install eventmachine json daemons
\end{verbatim}

\textbf{Operation instructions} \\
To run FSServer as a normal process, listen on \texttt{localhost:3000} and print the log messages to standard out run this command:
\begin{verbatim}
	./fsserver
\end{verbatim}

You can pass the following options to fsserver:
\begin{verbatim}
    --host=HOST (default is '0.0.0.0')
    --port=PORT (default is 3000)
    --timeout=TIMEOUT  (default is 30)
    --logfile=LOGFILE  (default is stdout)
\end{verbatim}

To run the server as a daemon run this command:
\begin{verbatim}
	./fsdaemon start
\end{verbatim}

options to daemon mode are given after an extra double dash like this:
\begin{verbatim}
	./fsdaemon start -- --option1=VALUE1 --options2=VALUE
\end{verbatim}

\section{FSClient}
\textbf{'Installation instructions on Ubuntu} \\
The fsclient needs ruby, the eventmachine gem and the JSON gem. To install run this command:
\begin{verbatim}
	sudo apt-get install ruby1.8 ruby1.8-dev rubygems1.8
	sudo gems install eventmachine json
\end{verbatim}

It is convenient to add the path of the fsclient executable to the PATH variable and a requirement to do so on the Ground Station in order for the upload script to work properly.

\textbf{Operation instructions} \\
Fsclient's help:
\begin{verbatim}
Usage: fsclient [options] <command [command_args ... ]>
    --host=HOST                  Server host (default is 0.0.0.0)
    --port=PORT                  Server port (default is 3000)
    --token=TOKEN                Token
    --timeout=SEC                Timeout option to command
    -i, --interactive            Interactive mode
    -d, --data-only              Only print data parameter
    -a, --auto-lock              Auto lock in interactive mode
    -n, --no-response            No-response option to command

\end{verbatim}

\section{FSClient}
\textbf{Installation instructions} \\
Install a Java Virtual Machine for your system.

\textbf{Operation instructions} \\
Double click the executable jarfile

\section{Upload script}
\textbf{Installation instructions} \\
Needs ruby, the JSON gem and a working fsclient that has been added to the PATH environment variable.

\textbf{Operation instructions} \\
upload\_file's help:
\begin{verbatim}
Usage: upload_file token filepath address
Description: Upload a file to an address in the satellites memory
Arguments:
    filepath (string)
    address (hexadecimal)
\end{verbatim}
